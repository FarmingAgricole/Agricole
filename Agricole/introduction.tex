\section{Introduction}
\label{sec:introduction}

Nous sommes trois �tudiants en deuxi�me ann�e de licence informatique � l'universit� de Cergy-Pontoise, Site St Martin. Plusieurs sujets nous ont �t� propos� , parmi ces derniers, apr�s lecture de chaque sujet, nous �tions tr�s attir�s par les projets de type jeux vid�o car ils �taient une grande 
part des loisirs de notre jeunesse. C'est pourquoi nous avons choisi le sujet Agricole, ce-dernier est un jeu de simulation d'une ferme agricole o� le joueur incarne le r�le du fermier et peut donc effectuer diverses t�ches quotidienne (tour en jeu), le but �tant de gagner de l'or via les ressources generer par la ferme, pour ensuite am�liorer cette-derniere. C'est par curiosit� que nous avons choisi ce sujet, en effet chaque membre du groupe a d�j� jouer � des jeux de ce genre nous trouvions donc int�ressant de savoir ce qui se trouvait derri�re ces interfaces graphiques. 

\paragraph{Fonctionnalit�s} Fonctionnalit�s du programme:

Nous avons d�cid�s que les �l�ments suivants seront pr�d�finis et non modifiables par
l'utilisateur:
\begin{itemize}
\item La taille de la carte de la ferme (map).
\item L'emplacement des parcelles, leurs superficies et capacit�s de production.
\item Le nombre de garage et emplacements de v�hicule.
\item Le march� (prix, contenue...).
\end{itemize}


